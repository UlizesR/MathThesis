\documentclass[12pt]{article}
\usepackage{setspace}      % For double spacing
% \usepackage{mathptmx}      % Times New Roman font
\usepackage{amsmath, amssymb, amsthm, amsfonts}
\usepackage{physics}
\usepackage[hidelinks]{hyperref}
\usepackage{color}
\usepackage{subcaption}
\usepackage{tikz}
\usepackage{pgfplots}
\usepackage{geometry}
\usepackage{enumitem}
\pgfplotsset{compat=1.18}

\geometry{margin=1in}

% Set double spacing
\setstretch{1.5}

\newcommand{\todo}[1]{\textcolor{red}{#1}}
\newcommand{\note}[1]{\textcolor{blue}{#1}}

\newcommand{\R}{\mathbb{R}}
\newcommand{\C}{\mathbb{C}}
\newcommand{\Z}{\mathbb{Z}}
\newcommand{\N}{\mathbb{N}}
\newcommand{\Q}{\mathbb{Q}}
\newcommand{\F}{\mathbb{F}}

\newtheorem{definition}{Definition}[section]

\title{Geodesics on Surfaces of Revolution and Ruled Surfaces: A Study of Symmetries and Isometries}
\author{Ulizes Raudales}
\date{\today}

\begin{document}
\maketitle

\newpage
\tableofcontents
\newpage

\begin{abstract}
    Geodesics, the "straightest" paths between points, are fundamental objects in geometry and physics. 
    While on flat surfaces geodesics are simply straight lines, on curved surfaces they can assume intricate forms. 
    In this paper, we examine geodesics on two distinct classes of surfaces: surfaces of revolution, which are generated by rotating a curve about an axis, and ruled surfaces, which are formed by translating a straight line along a prescribed path. 
    By leveraging the inherent symmetries of these surfaces and exploring the role of isometries—transformations that preserve distances—we aim to simplify the process of determining geodesics and illuminate how these symmetries influence geodesic behavior. 
\end{abstract}

\newpage

\section{Introduction}

It is well known that the shortest path between two points in the Euclidean plane is a straight line. 
This fact stems from the flatness of the plane, where the straight line minimizes distance. 
However, on curved surfaces—such as a sphere or a hyperboloid—the shortest path is not a Euclidean straight line but a curve known as a geodesic.

Geodesics are the natural extension of straight lines to curved spaces. 
They not only measure the intrinsic distance between points on a surface but also capture the surface’s geometry. 
In physics, geodesics describe the trajectories of particles moving under the influence of gravity and are often interpreted as paths of least action or resistance.

Determining geodesics, however, is typically challenging. 
In general, the geodesic equations form a system of nonlinear second-order differential equations whose solutions are rarely expressible in closed form. 
In this work, we focus on two special classes of surfaces: surfaces of revolution and ruled surfaces. 
The inherent symmetries of these surfaces lead to conserved quantities that simplify the geodesic equations, often reducing them to first-order relations that can be integrated explicitly.

Our main goal is to study the geodesics on surfaces of revolution and ruled surfaces and to explore how symmetries and isometries govern their behavior. 
We begin by introducing the fundamental concepts of differential geometry related to surfaces and geodesics. 
Then, we derive the geodesic equations for these classes of surfaces and show how symmetry properties—such as the cyclicity of the angular coordinate on a surface of revolution—yield conserved quantities (e.g., Clairaut’s relation) that simplify the analysis. 
Finally, we investigate the role of isometries in preserving geodesics and discuss how the intrinsic geometry of a surface influences the behavior of its geodesics.

This refined approach not only makes the differential equations more tractable but also provides clear geometric insight into the structure and properties of geodesics on curved surfaces.

\section{Background and Preliminaries}


\subsection{Surfaces in \texorpdfstring{$\R^3$}{R3}}

In differential geometry, a \emph{manifold} is an $n$-dimensional topological space that is locally homeomorphic to Euclidean space $\R^n$. In this work, we focus on surfaces, which are 2-dimensional manifolds embedded in $\R^3$. Such an embedding endows the surface with additional geometric structure.

\begin{definition}[Surface]
Let $S$ be a 2-dimensional manifold embedded in $\R^3$. Then there exists a smooth parametrization
\[
    \mathbf{r}: U \subset \R^2 \to S,
\]
where for each $(u,v) \in U$, 
\[
    \mathbf{r}(u,v) = \begin{bmatrix} x(u,v) \\ y(u,v) \\ z(u,v) \end{bmatrix}
\]
defines a point in $S$. In component (tensor) notation, one may also write
\[
    \mathbf{r}^1 = x(u,v), \quad \mathbf{r}^2 = y(u,v), \quad \mathbf{r}^3 = z(u,v).
\]
\end{definition}

The \emph{tangent plane} at a point $p=\mathbf{r}(u_0,v_0)$ is spanned by the partial derivatives of $\mathbf{r}$:
\[
    \mathbf{r}_u = \frac{\partial \mathbf{r}}{\partial u}(u_0,v_0) \quad \text{and} \quad \mathbf{r}_v = \frac{\partial \mathbf{r}}{\partial v}(u_0,v_0).
\]
Thus, the tangent space is defined as
\[
    T_{p}(S) = \operatorname{span}\{\mathbf{r}_u,\, \mathbf{r}_v\}.
\]

\begin{definition}[Tangent Vector]
A \textbf{tangent vector} $T \in T_{p}(S)$ at $p \in S$ is any vector that can be expressed as a linear combination of $\mathbf{r}_u$ and $\mathbf{r}_v$:
\[
    T = a\, \mathbf{r}_u + b\, \mathbf{r}_v, \quad \text{with } a,b\in \R.
\]
\end{definition}

The normal vector to the surface is obtained via the cross product of the tangent vectors:
\[
    \mathbf{N} = \mathbf{r}_u \times \mathbf{r}_v.
\]
This normal vector is orthogonal to the tangent plane and typically chosen to point in the outward direction.

\subsubsection{The Metric on a Surface}

A key element of the surface geometry is the metric, a smoothly varying inner product on the tangent spaces, which allows us to measure lengths, angles, and areas. In local coordinates $(u,v)$, the metric (or first fundamental form) is given by
\[
    g_{\mu\nu} = \mathbf{r}_\mu \cdot \mathbf{r}_\nu, \quad \mu,\nu \in \{u,v\},
\]
and can be represented by the matrix
\[
    g = \begin{bmatrix} E & F \\ F & G \end{bmatrix},
\]
where
\[
    E = \mathbf{r}_u \cdot \mathbf{r}_u,\quad F = \mathbf{r}_u \cdot \mathbf{r}_v,\quad G = \mathbf{r}_v \cdot \mathbf{r}_v.
\]

For an infinitesimal displacement
\[
    d\mathbf{q} = \begin{bmatrix} du \\ dv \end{bmatrix},
\]
the first fundamental form is written as
\[
    ds^2 = g_{\mu\nu}\, dq^\mu dq^\nu = E\, du^2 + 2F\, du\, dv + G\, dv^2.
\]

\subsubsection{Curves on Surfaces}

With the metric defined, we can now study curves on the surface.

\begin{definition}[Curve]
A \textbf{curve} $\gamma$ on a surface $S$ is a smooth map
\[
    \gamma : I \to S,
\]
where $I \subset \R$ is an interval and for each $t\in I$, $\gamma(t)$ is a point on $S$.
The curve is parametrized by the coordinates $(u(t), v(t))$ in the local chart, so we can write
\[
    \gamma(t) = \mathbf{r}(u(t), v(t)) = \begin{bmatrix} x(u(t), v(t)) \\ y(u(t), v(t)) \\ z(u(t), v(t)) \end{bmatrix}.
\]
\end{definition}

The length of a curve and the angles between curves are determined by the metric. 
We will want to focus on unit speed curves, which are curves with constant speed equal to one.
\begin{definition}[Unit Speed Curve]
A \textbf{unit speed curve} $\gamma(t)$ is a curve such that the length of the tangent vector is constant and equal to one:
\[
    \abs{\dot{\gamma}(t)} = \dot{\gamma}(t) \cdot \dot{\gamma}(t) = 1.
\]
\end{definition}

We will be working with unit speed curves for the most part, as they simplify the calculations and allow us to focus on the geometric properties of the curves without worrying about their speed.

\subsubsection{Geodesics}
Geodesics are curves that locally minimize distance on a surface.

\begin{definition}[Geodesic]
A \textbf{geodesic} $\gamma(t)$ on a surface $S$ is a curve that minimizes the distance between two points on the surface.
This is equivalent to saying that the geodesic is a critical point of the length functional
\[
    L[\gamma] = \int_{t_0}^{t_1} \abs{\dot{\gamma}(t)} dt,
\]
where $\dot{\gamma}(t)$ is the tangent vector to the curve at time $t$.
\end{definition}

\subsection{Surfaces of revolution}

Surfaces of revolution are a class of surfaces obtained by rotating a curve about an axis.

\begin{definition}[Surface of Revolution]
A \textbf{surface of revolution} is a surface obtained by rotating a ($C^{2}$ smooth) curve \(\gamma(u) = (x(u), y(u))\) about an axis in the plane for which the curve lies.
The curve \(\gamma(u)\) is called the profile curve of the surface.
\end{definition}

We will be focusing on surfaces of revolution with rotational symmetry about the \(z\)-axis.
So our parametric representation of the surface will take the form
$$
    \mathbf{r}(u, v) = \begin{bmatrix} x(u) \cos (v) \\ x(u) \sin (v) \\ z(u) \end{bmatrix},
$$
where \(x(u)\) and \(z(u)\) are smooth functions that define the curve of revolution.

The parameter \(u\) will be the parameter along the profile curve, while \(v\) will be the angle of rotation about the \(z\)-axis.
The metric for this surface can be computed as follows:
\[
    ds^{2} = (\dot{x}(u)^{2} + \dot{z}(u)^{2}) du^{2} + x(u)^{2} dv^{2},
\]
where
\begin{align*}
    E &= \mathbf{r}_u \cdot \mathbf{r}_u = x'(u)^2 + z'(u)^2,\\
    F &= \mathbf{r}_u \cdot \mathbf{r}_v = 0,\\
    G &= \mathbf{r}_v \cdot \mathbf{r}_v = x(u)^2.
\end{align*}

Given we are working with units speed curves, $E = 1$, we can write the metric as
\[
    ds^{2} = du^{2} + x(u)^{2} dv^{2}.
\]

The metric tensor can be expressed in matrix form as
\[
    g = \begin{bmatrix} 1 & 0 \\ 0 & x(u)^{2} \end{bmatrix}.
\]
with the inverse metric given by
\[
    g^{-1} = \begin{bmatrix} 1 & 0 \\ 0 & \frac{1}{x(u)^{2}} \end{bmatrix}.
\]

\subsection{Clairaut’s Relation}
Clairaut's relation is a powerful result that arises from the symmetry of the surface of revolution.
It states that for any curve on the surface that satisfy the following conditions:
\[
    R \sin(\phi) = \text{constant}
\]
where \(R\) is the distance from the axis of rotation and \(\phi\) is the angle between the curve and the axis of rotation.
If this condition is satisfied, then the curve is a geodesic on the surface of revolution.

\section{Geodesics on Surfaces}

Now that we have established the necessary background, we can proceed to study the shortest paths on surfaces, known as geodesics.
These curves are fundamental in geometry and physics, providing insights into the intrinsic geometry of surfaces and the behavior of particles moving on them.

\begin{definition}[Geodesic]
    A \textbf{geodesic} on a surface $S$ is a curve $\gamma$ that locally minimizes the length between two points on the surface.
\end{definition}

The geodesic equation is a second-order differential equation that describes the geodesics on a surface.
It can be derived by minimizing the length functional associated with the surface.

\subsection{Deriving the Geodesic Equations}

To derive the geodesic equations, we begin by minimizing the Lagrangian associated with the surface.
Let $\mathcal{L} = g_{\mu\nu} \dot{q}^\mu \dot{q}^\nu$ be the Lagrangian, where $g_{\mu\nu}$ is the metric tensor and $\dot{q}^\mu = \dv{q^\mu}{t}$.
The action functional is then
\[
    S[\gamma] = \int_{a}^{b} \mathcal{L} \dd{t} = \int_{a}^{b} g_{\mu\nu} \dot{q}^\mu \dot{q}^\nu \dd{t}.
\]
The Euler-Lagrange equations for this Lagrangian are
\[
    \dv{t} \left( \pdv{\mathcal{L}}{\dot{q}^\lambda} \right) - \pdv{\mathcal{L}}{q^\lambda} = 0.
\]
Substituting $\mathcal{L} = g_{\mu\nu} \dot{q}^\mu \dot{q}^\nu$ into the Euler-Lagrange equations, we obtain
\begin{align*}
    \pdv{\mathcal{L}}{q^{\lambda}} &= \pdv{g_{\mu\nu}}{q^{\lambda}} \dot{q}^{\mu} \dot{q}^{\nu}  \\
    \pdv{\mathcal{L}}{\dot{q}^{\lambda}} &= g_{\mu\nu} \left( \pdv{\dot{q}^{\mu}}{q^{\lambda}} \dot{q}^{\nu} + \dot{q}^{\mu} \pdv{\dot{q}^{\nu}}{q^{\lambda}} \right) = 2 g_{\mu\nu} \dot{q}^{\mu} \delta_{\lambda}^{\nu} = 2 g_{\mu\lambda} \dot{q}^{\mu} \\
    \dv{t} \left( \pdv{\mathcal{L}}{\dot{q}^{\lambda}} \right) &= 2 \dv{g_{\mu\lambda}}{t} \dot{q}^{\mu} + 2 g_{\mu\lambda} \ddot{q}^{\mu} \\
\end{align*}

We note that the metric tensor is symmetric, so $g_{\mu\nu} = g_{\nu\mu}$.
So the derivative of the metric tensor with respect to time can be written as
\[
\dv{g_{\mu\lambda}}{t} = \pdv{g_{\mu\lambda}}{q^{\nu}} \dot{q}^{\nu} 
\]
So 
\[
\dv{t} \left( \pdv{\mathcal{L}}{\dot{q}^{\lambda}} \right) = 2 \left(\pdv{g_{\mu\lambda}}{q^{\nu}} \dot{q}^{\nu} \dot{q}^{\mu} + g_{\mu\lambda} \ddot{q}^{\mu} \right)
\] 
Substituting these expressions into the Euler-Lagrange equations, we obtain
\[
2 \left(\pdv{g_{\mu\lambda}}{q^{\nu}} \dot{q}^{\nu} \dot{q}^{\mu} + g_{\mu\lambda} \ddot{q}^{\mu} \right) - \pdv{g_{\mu\nu}}{q^{\lambda}} \dot{q}^{\mu} \dot{q}^{\nu} = 0.
\]
We rewrite this equation as
\[
g_{\mu\lambda} \ddot{q}^{\mu} + \frac{1}{2} \left(2\pdv{g_{\mu\lambda}}{q^{\nu}} -  \pdv{g_{\mu\nu}}{q^{\lambda}} \right) \dot{q}^{\mu} \dot{q}^{\nu} = 0
\]
Given that the metric tensor is symmetric, we can rewrite this equation as
\[
g_{\mu\lambda} \ddot{q}^{\mu} + \frac{1}{2} \left( \pdv{g_{\mu\lambda}}{q^{\nu}} + \pdv{g_{\nu\lambda}}{q^{\mu}} - \pdv{g_{\mu\nu}}{q^{\lambda}} \right) \dot{q}^{\mu} \dot{q}^{\nu} = 0.
\]
We multiply both sides by \(g^{\lambda\sigma}\) to obtain
\[
g^{\lambda\sigma} g_{\mu\lambda} \ddot{q}^{\mu} + \frac{1}{2} g^{\lambda\sigma} \left( \pdv{g_{\mu\lambda}}{q^{\nu}} + \pdv{g_{\nu\lambda}}{q^{\mu}} - \pdv{g_{\mu\nu}}{q^{\lambda}} \right) \dot{q}^{\mu} \dot{q}^{\nu} = 0.
\]
We recognize that \(g^{\lambda\sigma} g_{\mu\lambda} = \delta_{\mu}^{\sigma}\), so the first term simplifies to \(\ddot{q}^{\sigma}\).
Thus, the geodesic equation on a surface is
\[
\ddot{q}^{\sigma} + \frac{1}{2} g^{\lambda\sigma} \left( \pdv{g_{\mu\lambda}}{q^{\nu}} + \pdv{g_{\nu\lambda}}{q^{\mu}} - \pdv{g_{\mu\nu}}{q^{\lambda}} \right) \dot{q}^{\mu} \dot{q}^{\nu} = 0.
\]
We introduce the Christoffel symbols \(\Gamma_{\mu\nu}^{\sigma}\) by defining
\[
   \Gamma_{\mu\nu}^{\sigma} = \frac{1}{2} g^{\lambda\sigma} \left( \pdv{g_{\mu\lambda}}{q^{\nu}} + \pdv{g_{\nu\lambda}}{q^{\mu}} - \pdv{g_{\mu\nu}}{q^{\lambda}} \right).
\]
The geodesic equation can then be written as
\begin{equation}
    \ddot{q}^{\sigma} + \Gamma_{\mu\nu}^{\sigma} \dot{q}^{\mu} \dot{q}^{\nu} = 0.
\end{equation}
In matrix form, the geodesic equation is
\[
\begin{bmatrix}
\ddot{q}^{1} \\
\ddot{q}^{2} \\
\end{bmatrix} + \begin{bmatrix}
\begin{bmatrix}
\Gamma_{11}^{1} & \Gamma_{12}^{1} \\
\Gamma_{21}^{1} & \Gamma_{22}^{1} \\
\end{bmatrix} & \begin{bmatrix}
\Gamma_{11}^{2} & \Gamma_{12}^{2} \\
\Gamma_{21}^{2} & \Gamma_{22}^{2} \\
\end{bmatrix}
\end{bmatrix} \begin{bmatrix}
\dot{q}^{1} \\
\dot{q}^{2} \\
\end{bmatrix} \begin{bmatrix}
\dot{q}^{1} \\
\dot{q}^{2} \\
\end{bmatrix} = 0.
\]

\subsubsection{The Christoffel Symbols}
A direct computation shows that:
\[
\begin{aligned}
\Gamma^1_{11} &= \frac{1}{2D}\Bigl[G\,E_u - F\bigl(2F_u - E_v\bigr)\Bigr],\\[1mm]
\Gamma^1_{12} = \Gamma^1_{21} &= \frac{1}{2D}\Bigl[G\,E_v - F\,G_u\Bigr],\\[1mm]
\Gamma^1_{22} &= \frac{1}{2D}\Bigl[G\bigl(2F_v - G_u\bigr) - F\,G_v\Bigr],\\[1mm]
\Gamma^2_{11} &= \frac{1}{2D}\Bigl[E\bigl(2F_u - E_v\bigr) - F\,E_u\Bigr],\\[1mm]
\Gamma^2_{12} = \Gamma^2_{21} &= \frac{1}{2D}\Bigl[E\,G_u - F\,E_v\Bigr],\\[1mm]
\Gamma^2_{22} &= \frac{1}{2D}\Bigl[E\,G_v - F\bigl(2F_v - G_u\bigr)\Bigr],
\end{aligned}
\]
where \(D = EG - F^2\).
We note that the Christoffel symbols is not a tensor, but it is a connection that allows us to differentiate vectors along a curve on the surface.

The geodesic equations in these coordinates are given by
\[
\ddot{x}^k + \Gamma^k_{ij}\,\dot{x}^i\,\dot{x}^j = 0,\quad k=1,2,
\]
where we set \(x^1=u\) and \(x^2=v\). Thus, the geodesic equations for each component are:
\begin{equation}
    \ddot{u} + \Gamma^1_{11}\,\dot{u}^2 + 2\,\Gamma^1_{12}\,\dot{u}\dot{v} + \Gamma^1_{22}\,\dot{v}^2 = 0
\end{equation}
\begin{equation}
    \ddot{v} + \Gamma^2_{11}\,\dot{u}^2 + 2\,\Gamma^2_{12}\,\dot{u}\dot{v} + \Gamma^2_{22}\,\dot{v}^2 = 0
\end{equation}


\subsection{Geodesics on Surfaces of Revolution}

We now apply the geodesic equations to surfaces of revolution, which are generated by rotating a curve about an axis.
We will focus on surfaces of revolution with rotational symmetry about the \(z\)-axis, which can be parametrized as
\[
\mathbf{r}(u, v) = \begin{bmatrix} x(v) \cos (u) \\ x(v) \sin (u) \\ z(v) \end{bmatrix}.
\]
The metric tensor for a unit speed curve on this surface is given by
\[
g = \begin{bmatrix} 1 & 0 \\ 0 & x^2 \end{bmatrix},
\]
where $E = 1$, $F = 0$, and $G = x^2$.
Since we are 

Taking the derivatives of these terms with respect to the coordinates, we find
\[
\begin{aligned}
E_u &= 0, &\quad E_v &= 0 \\
F_u &= 0, &\quad F_v &= 0 \\
G_u &= 2x\,x_{u} &\quad G_v &= 0
\end{aligned}
\]

So the inverse metric tensor is
\[
g^{-1} = \begin{bmatrix} 1 & 0 \\[4mm] 0 & \frac{1}{x^2} \end{bmatrix},
\]

The Christoffel symbols for this surface are
\[
\begin{aligned}
    \Gamma_{11}^{1} &= \frac{E_{u}}{2E} = 0 &\quad \Gamma_{12}^{1} &= \Gamma_{21}^{1} = \frac{E_{v}}{2E} = 0 &\quad \Gamma_{22}^{1} &= -\frac{G_{u}}{2E} = -x\,x_{u} \\[4mm]
    \Gamma_{11}^{2} &= -\frac{E_{v}}{2G} = 0 &\quad \Gamma_{12}^{2} &= \Gamma_{21}^{2} = \frac{G_{u}}{2G} = \frac{x_{u}}{x} &\quad \Gamma_{22}^{2} &= -\frac{E_{u}}{2E} = 0.
\end{aligned}
\]

The geodesic equations for this surface are
\begin{equation}
    \ddot{u} - x\,x_{u}\,\dot{v}^2 = 0,
\end{equation}
\begin{equation}
    \ddot{v} + 2\frac{x_{u}}{x}\,\dot{u}\dot{v} = 0.
\end{equation}

The second equation can be solved by separating variables:
\[
    \frac{\ddot{v}}{\dot{v}} = -2\frac{x_{u}}{x}\dot{u}
\]
Integrating both sides, we obtain
\[
    \ln(\dot{v}) = -2 \ln(x(u)) + C_1,
\]
where \(C_1\) is a constant of integration.
Exponentiating both sides, we find
\[
    \dot{v} = \frac{C_2}{x^2}
\]
where \(C_2 = e^{C_1}\) is a constant.
Multiplying both sides by \(x^2\), we have
\[
    x^2 \dot{v} = C_2.
\]
This is a conserved quantity, which means that the quantity \(x^2 \dot{v}\) is constant along the geodesic.


\end{document}