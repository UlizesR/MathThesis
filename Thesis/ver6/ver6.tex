\documentclass[12pt]{article}
\usepackage{setspace}      % For double spacing
% \usepackage{mathptmx}      % Times New Roman font
\usepackage{amsmath, amssymb, amsthm, amsfonts}
\usepackage{physics}
\usepackage[hidelinks]{hyperref}
\usepackage{color}
\usepackage{subcaption}
\usepackage{tikz}
\usepackage{pgfplots}
\usepackage{geometry}
\usepackage{enumitem}
\pgfplotsset{compat=1.18}

\geometry{margin=1in}

% Set double spacing
\setstretch{1.5}

\newcommand{\todo}[1]{\textcolor{red}{#1}}
\newcommand{\note}[1]{\textcolor{blue}{#1}}

\newcommand{\R}{\mathbb{R}}
\newcommand{\C}{\mathbb{C}}
\newcommand{\Z}{\mathbb{Z}}
\newcommand{\N}{\mathbb{N}}
\newcommand{\Q}{\mathbb{Q}}
\newcommand{\F}{\mathbb{F}}

\newtheorem{definition}{Definition}[section]

\title{Geodesics, Symmetries, and Isometries: A Comprehensive Study and Applications}
\author{Ulizes Raudales}
\date{\today}

\begin{document}
\maketitle

\newpage
\tableofcontents
\newpage

\begin{abstract}
    Geodesics, the "straightest" paths between points, are fundamental objects in geometry and physics. 
    While on flat surfaces geodesics are simply straight lines, on curved surfaces they can assume intricate forms. 
    In this paper, we examine geodesics on two distinct classes of surfaces: surfaces of revolution, which are generated by rotating a curve about an axis, and ruled surfaces, which are formed by translating a straight line along a prescribed path. 
    By leveraging the inherent symmetries of these surfaces and exploring the role of isometries—transformations that preserve distances—we aim to simplify the process of determining geodesics and illuminate how these symmetries influence geodesic behavior. 
\end{abstract}

\newpage

\section*{Notation and Conventions}
\begin{itemize}
    \item Einstein summation is assumed (repeated indices are summed over).
    \item Newtonian notation is used for time derivatives: \(\dot{q} = \frac{dq}{dt}\) and \(\ddot{q} = \frac{d^2q}{dt^2}\).
    \item Denote partial derivatives by \(q_u = \frac{\partial q}{\partial u}\) and \(q_v = \frac{\partial q}{\partial v}\).
    \item Use \(q^1=u\), \(q^2=v\) (or generally \(q^\mu\) for \(\mu=1,2\)) for surface coordinates.
    \item Boldface letters (e.g., \(\mathbf{r}\)) denote vectors.
\end{itemize}

\newpage

\section{Introduction}

It is well known that the shortest path between two points in the Euclidean plane is a straight line. 
This fact stems from the flatness of the plane, where the straight line minimizes distance. 
However, on curved surfaces—such as a sphere or a hyperboloid—the shortest path is not a Euclidean straight line but a curve known as a geodesic.

Geodesics are the natural extension of straight lines to curved spaces. 
They not only measure the intrinsic distance between points on a surface but also capture the surface’s geometry. 
In physics, geodesics describe the trajectories of particles moving under the influence of gravity and are often interpreted as paths of least action or resistance.

Determining geodesics, however, is typically challenging. 
In general, the geodesic equations form a system of nonlinear second-order differential equations whose solutions are rarely expressible in closed form. 
In this work, we focus on two special classes of surfaces: surfaces of revolution and ruled surfaces. 
The inherent symmetries of these surfaces lead to conserved quantities that simplify the geodesic equations, often reducing them to first-order relations that can be integrated explicitly.

Our main goal is to study the geodesics on surfaces of revolution and ruled surfaces and to explore how symmetries and isometries govern their behavior. 
We begin by introducing the fundamental concepts of differential geometry related to surfaces and geodesics. 
Then, we derive the geodesic equations for these classes of surfaces and show how symmetry properties—such as the cyclicity of the angular coordinate on a surface of revolution—yield conserved quantities (e.g., Clairaut’s relation) that simplify the analysis. 
Finally, we investigate the role of isometries in preserving geodesics and discuss how the intrinsic geometry of a surface influences the behavior of its geodesics.

This refined approach not only makes the differential equations more tractable but also provides clear geometric insight into the structure and properties of geodesics on curved surfaces.

\section{Background and Preliminaries}


\subsection{Surfaces in \texorpdfstring{$\R^3$}{R3}}

In differential geometry, a \emph{manifold} is an $n$-dimensional topological space that is locally homeomorphic to Euclidean space $\R^n$. In this work, we focus on surfaces, which are 2-dimensional manifolds embedded in $\R^3$. Such an embedding endows the surface with additional geometric structure.

\begin{definition}[Surface]
Let $S$ be a 2-dimensional manifold embedded in $\R^3$. Then there exists a smooth parametrization
\[
    \mathbf{r}: U \subset \R^2 \to S,
\]
where for each $(u,v) \in U$, 
\[
    \mathbf{r}(u,v) = \begin{bmatrix} x(u,v) \\ y(u,v) \\ z(u,v) \end{bmatrix}
\]
defines a point in $S$. In component (tensor) notation, one may also write
\[
    \mathbf{r}^1 = x(u,v), \quad \mathbf{r}^2 = y(u,v), \quad \mathbf{r}^3 = z(u,v).
\]
\end{definition}

The \emph{tangent plane} at a point $p=\mathbf{r}(u_0,v_0)$ is spanned by the partial derivatives of $\mathbf{r}$:
\[
    \mathbf{r}_u = \frac{\partial \mathbf{r}}{\partial u}(u_0,v_0) \quad \text{and} \quad \mathbf{r}_v = \frac{\partial \mathbf{r}}{\partial v}(u_0,v_0).
\]
Thus, the tangent space is defined as
\[
    T_{p}(S) = \operatorname{span}\{\mathbf{r}_u,\, \mathbf{r}_v\}.
\]

\begin{definition}[Tangent Vector]
A \textbf{tangent vector} $T \in T_{p}(S)$ at $p \in S$ is any vector that can be expressed as a linear combination of $\mathbf{r}_u$ and $\mathbf{r}_v$:
\[
    T = a\, \mathbf{r}_u + b\, \mathbf{r}_v, \quad \text{with } a,b\in \R.
\]
\end{definition}

The normal vector to the surface is obtained via the cross product of the tangent vectors:
\[
    \mathbf{N} = \mathbf{r}_u \times \mathbf{r}_v.
\]
This normal vector is orthogonal to the tangent plane and typically chosen to point in the outward direction.

\subsubsection{The Metric on a Surface}

A key element of the surface geometry is the metric, a smoothly varying inner product on the tangent spaces, which allows us to measure lengths, angles, and areas. In local coordinates $(u,v)$, the metric (or first fundamental form) is given by
\[
    g_{\mu\nu} = \mathbf{r}_\mu \cdot \mathbf{r}_\nu, \quad \mu,\nu \in \{u,v\},
\]
and can be represented by the matrix
\[
    g = \begin{bmatrix} E & F \\ F & G \end{bmatrix},
\]
where
\[
    E = \mathbf{r}_u \cdot \mathbf{r}_u,\quad F = \mathbf{r}_u \cdot \mathbf{r}_v,\quad G = \mathbf{r}_v \cdot \mathbf{r}_v.
\]

For an infinitesimal displacement
\[
    d\mathbf{q} = \begin{bmatrix} du \\ dv \end{bmatrix},
\]
the first fundamental form is written as
\[
    ds^2 = g_{\mu\nu}\, dq^\mu dq^\nu = E\, du^2 + 2F\, du\, dv + G\, dv^2.
\]

\subsubsection{Curves on Surfaces}

With the metric defined, we can now study curves on the surface.

\begin{definition}[Curve]
A \textbf{curve} $\gamma$ on a surface $S$ is a smooth map
\[
    \gamma : I \to S,
\]
where $I \subset \R$ is an interval and for each $t\in I$, $\gamma(t)$ is a point on $S$.
The curve is parametrized by the coordinates $(u(t), v(t))$ in the local chart, so we can write
\[
    \gamma(t) = \mathbf{r}(u(t), v(t)) = \begin{bmatrix} x(u(t), v(t)) \\ y(u(t), v(t)) \\ z(u(t), v(t)) \end{bmatrix}.
\]
\end{definition}

The length of a curve and the angles between curves are determined by the metric. 

\begin{definition}[Unit Speed Curve]
A \textbf{unit speed curve} is a curve $\gamma(t)$ such that the tangent vector $\dot{\gamma}(t)$ has constant length:
\[
    \abs{\dot{\gamma}(t)} = \sqrt{g_{\mu\nu} \dot{q}^\mu \dot{q}^\nu} = E \dot{u}^2 + 2F \dot{u} \dot{v} + G \dot{v}^2 = 1.
\]
\end{definition}

Under this condition, the curve is said to be parametrized by arc length.
The simplification of the geodesic equations is often achieved by considering unit speed curves, as they yield a more straightforward relationship between the curve's geometry and its parametrization.

\subsection{Surfaces of revolution}

Surfaces of revolution are a class of surfaces obtained by rotating a curve about an axis.

\begin{definition}[Surface of Revolution]
A \textbf{surface of revolution} is a surface obtained by rotating a ($C^{2}$ smooth) curve \(\gamma(u) = (x(u), y(u))\) about an axis in the plane for which the curve lies.
The curve \(\gamma(u)\) is called the profile curve of the surface.
\end{definition}

We will be focusing on surfaces of revolution with rotational symmetry about the \(z\)-axis.
So our parametric representation of the surface will take the form
\begin{equation} \label{eq:surface_of_revolution}
    \mathbf{r}(u, v) = \begin{bmatrix} x(u) \cos (v) \\ x(u) \sin (v) \\ z(u) \end{bmatrix},
\end{equation}
where \(x(u)\) and \(z(u)\) are smooth functions that define the curve of revolution.

The parameter \(u\) will be the parameter along the profile curve, while \(v\) will be the angle of rotation about the \(z\)-axis.

Finding the components of the metric tensor for this surface is straightforward.
The tangent vectors are given by
\[
    \mathbf{r}_u = \begin{bmatrix} x_u \cos (v) \\ x_u \sin (v) \\ z_u \end{bmatrix}, \quad \mathbf{r}_v = \begin{bmatrix} -x(u) \sin (v) \\ x(u) \cos (v) \\ 0 \end{bmatrix}.
\]
The metric tensor is then given by
\[
    g_{\mu\nu} = \begin{bmatrix} x_{u}^2 + z_u^2 & 0 \\ 0 & x^2 \end{bmatrix},
\]
where \(E = x_{u}^2 + z_u^2\), \(F = 0\), and \(G = x^2\).

So the first fundamental form is given by
\begin{equation}
    ds^2 = \left( x_{u}^2 + z_u^2 \right) du^2 + x^2 dv^2.
\end{equation}

\subsection{The Euler-Lagrange Equation}

Since geodesics act as the shortest paths between points on a surface, they can be think of as the curves that minimizes the action functional.
We borrow from variational calculus to find these geodesics by minimizing the action functional associated with the surface.

The action functional is defined as
\[
    S[\gamma] = \int_{a}^{b} L(t, q, \dot{q}) \dd{t},
\]
where \(L\) is the Lagrangian of the system and \(\gamma\) is the curve on the surface.

To find the minimizing curve, we begin by taking a small variation of the curve \(\gamma(t)\) by writing
\[
    \gamma(t) \to \gamma(t) + \epsilon \eta(t),
\]
where \(\eta(t)\) is a smooth function with \(\eta(a) = \eta(b) = 0\).

The derivative of the curve then changes as 
\[
    \dv{\gamma}{t} \to \dv{\gamma}{t} + \epsilon \dv{\eta}{t}.
\]
Under this variation, the Lagrangian becomes
\[
    L(t, q, \dot{q}) \to L(t, q + \epsilon \eta, \dot{q} + \epsilon \dot{\eta}).
\]
We then take the first-order expansion of the Lagrangian in \(\epsilon\):
\[
    L(t, q + \epsilon \eta, \dot{q} + \epsilon \dot{\eta}) = L(t, q, \dot{q}) + \epsilon \left( \pdv{L}{q} \eta + \pdv{L}{\dot{q}} \dv{\eta}{t} \right) + \mathcal{O}(\epsilon^2).    
\]
We ignore the higher-order terms and integrate the Lagrangian over the interval \([a,b]\) to obtain the action functional:
\[
    S[\gamma + \epsilon \eta] = \int_{a}^{b} L(t, q, \dot{q}) \dd{t} + \epsilon \int_{a}^{b} \left( \pdv{L}{q} \eta + \pdv{L}{\dot{q}} \dot{\eta} \right) \dd{t}
\]
The first variation of the action functional is then
\[
    \delta S = \dv{S}{\epsilon} \bigg|_{\epsilon = 0} = \int_{a}^{b} \left[ \pdv{L}{q} \eta + \pdv{L}{\dot{q}} \dot{\eta} \right] \dd{t}.
\]
For $\gamma$ to be a geodesic, the action functional must be stationary under all variations $\eta(t)$, so we require that $\delta S = 0$ for all $\eta(t)$.

We can simplify this expression by integrating by parts:
\[
    \delta S = \int_{a}^{b} \left[ \pdv{L}{q} - \dv{t} \left(\pdv{L}{\dot{q}}\right) \right] \eta \dd{t} + \left[ \pdv{L}{\dot{q}} \eta \right]_{a}^{b}.
\]
Since $\eta(a) = \eta(b) = 0$, the boundary terms vanish, and we are left with
\[
    \int_{a}^{b} \left(\left[ \pdv{L}{q} - \dv{t} \left(\pdv{L}{\dot{q}}\right) \right] \eta \right) \dd{t} = 0.
\]
Since $\eta$ is arbitrary (subject to vanishing at the endpoints), the integrand must vanish identically:
\[
    \pdv{L}{q} - \dv{t} \left(\pdv{L}{\dot{q}}\right) = 0.
\]
These are the Euler-Lagrange equations for the action functional, which describe the extremal curves that minimize the action.
The idea is to find the Lagrangian that describes the geodesics on the surface and then solve the Euler-Lagrange equations to determine the geodesics.

\section{Geodesics}

Now that we have established the necessary background, we can proceed to study the shortest paths on surfaces, known as geodesics.
These curves are fundamental in geometry and physics, providing insights into the intrinsic geometry of surfaces and the behavior of particles moving on them.

\begin{definition}[Geodesic]
    A \textbf{geodesic} on a surface $S$ is a curve $\gamma$ that locally minimizes the length between two points on the surface.
\end{definition}

Another property of geodesics is that they have constant speed \cite{Oprea2007}, That is, the tangent vector to the geodesic has constant length.
\[
    \abs{\dot{\gamma}(t)} = \text{constant}.
\]

The geodesic equation is a second-order differential equation that describes the geodesics on a surface.
It can be derived by minimizing the length functional associated with the surface.

\subsection{Deriving the Geodesic Equations}

To derive the geodesic equations, we begin by minimizing the Lagrangian associated with the surface.
Let $\mathcal{L} = g_{\mu\nu} \dot{q}^\mu \dot{q}^\nu$ be the Lagrangian, where $g_{\mu\nu}$ is the metric tensor and $\dot{q}^\mu = \dv{q^\mu}{t}$.
The action functional is then
\[
    S[\gamma] = \int_{a}^{b} \mathcal{L} \dd{t} = \int_{a}^{b} g_{\mu\nu} \dot{q}^\mu \dot{q}^\nu \dd{t}.
\]
The Euler-Lagrange equations for this Lagrangian are
\[
    \dv{t} \left( \pdv{\mathcal{L}}{\dot{q}^\lambda} \right) - \pdv{\mathcal{L}}{q^\lambda} = 0.
\]
Substituting $\mathcal{L} = g_{\mu\nu} \dot{q}^\mu \dot{q}^\nu$ into the Euler-Lagrange equations, we obtain
\begin{align*}
    \pdv{\mathcal{L}}{q^{\lambda}} &= \pdv{g_{\mu\nu}}{q^{\lambda}} \dot{q}^{\mu} \dot{q}^{\nu}  \\
    \pdv{\mathcal{L}}{\dot{q}^{\lambda}} &= g_{\mu\nu} \left( \pdv{\dot{q}^{\mu}}{q^{\lambda}} \dot{q}^{\nu} + \dot{q}^{\mu} \pdv{\dot{q}^{\nu}}{q^{\lambda}} \right) = 2 g_{\mu\nu} \dot{q}^{\mu} \delta_{\lambda}^{\nu} = 2 g_{\mu\lambda} \dot{q}^{\mu} \\
    \dv{t} \left( \pdv{\mathcal{L}}{\dot{q}^{\lambda}} \right) &= 2 \dv{g_{\mu\lambda}}{t} \dot{q}^{\mu} + 2 g_{\mu\lambda} \ddot{q}^{\mu} \\
\end{align*}

We note that the metric tensor is symmetric, so $g_{\mu\nu} = g_{\nu\mu}$.
So the derivative of the metric tensor with respect to time can be written as
\[
\dv{g_{\mu\lambda}}{t} = \pdv{g_{\mu\lambda}}{q^{\nu}} \dot{q}^{\nu} 
\]
So 
\[
\dv{t} \left( \pdv{\mathcal{L}}{\dot{q}^{\lambda}} \right) = 2 \left(\pdv{g_{\mu\lambda}}{q^{\nu}} \dot{q}^{\nu} \dot{q}^{\mu} + g_{\mu\lambda} \ddot{q}^{\mu} \right)
\] 
Substituting these expressions into the Euler-Lagrange equations, we obtain
\[
2 \left(\pdv{g_{\mu\lambda}}{q^{\nu}} \dot{q}^{\nu} \dot{q}^{\mu} + g_{\mu\lambda} \ddot{q}^{\mu} \right) - \pdv{g_{\mu\nu}}{q^{\lambda}} \dot{q}^{\mu} \dot{q}^{\nu} = 0.
\]
We rewrite this equation as
\[
g_{\mu\lambda} \ddot{q}^{\mu} + \frac{1}{2} \left(2\pdv{g_{\mu\lambda}}{q^{\nu}} -  \pdv{g_{\mu\nu}}{q^{\lambda}} \right) \dot{q}^{\mu} \dot{q}^{\nu} = 0
\]
Given that the metric tensor is symmetric, we can rewrite this equation as
\[
g_{\mu\lambda} \ddot{q}^{\mu} + \frac{1}{2} \left( \pdv{g_{\mu\lambda}}{q^{\nu}} + \pdv{g_{\nu\lambda}}{q^{\mu}} - \pdv{g_{\mu\nu}}{q^{\lambda}} \right) \dot{q}^{\mu} \dot{q}^{\nu} = 0.
\]
We multiply both sides by \(g^{\lambda\sigma}\) to obtain
\[
g^{\lambda\sigma} g_{\mu\lambda} \ddot{q}^{\mu} + \frac{1}{2} g^{\lambda\sigma} \left( \pdv{g_{\mu\lambda}}{q^{\nu}} + \pdv{g_{\nu\lambda}}{q^{\mu}} - \pdv{g_{\mu\nu}}{q^{\lambda}} \right) \dot{q}^{\mu} \dot{q}^{\nu} = 0.
\]
We recognize that \(g^{\lambda\sigma} g_{\mu\lambda} = \delta_{\mu}^{\sigma}\), so the first term simplifies to \(\ddot{q}^{\sigma}\).
Thus, the geodesic equation on a surface is
\[
\ddot{q}^{\sigma} + \frac{1}{2} g^{\lambda\sigma} \left( \pdv{g_{\mu\lambda}}{q^{\nu}} + \pdv{g_{\nu\lambda}}{q^{\mu}} - \pdv{g_{\mu\nu}}{q^{\lambda}} \right) \dot{q}^{\mu} \dot{q}^{\nu} = 0.
\]
We introduce the Christoffel symbols \(\Gamma_{\mu\nu}^{\sigma}\) by defining
\[
   \Gamma_{\mu\nu}^{\sigma} = \frac{1}{2} g^{\lambda\sigma} \left( \pdv{g_{\mu\lambda}}{q^{\nu}} + \pdv{g_{\nu\lambda}}{q^{\mu}} - \pdv{g_{\mu\nu}}{q^{\lambda}} \right).
\]
The geodesic equation can then be written as
\begin{equation}
    \ddot{q}^{\sigma} + \Gamma_{\mu\nu}^{\sigma} \dot{q}^{\mu} \dot{q}^{\nu} = 0.
\end{equation}

% \subsubsection{The Christoffel Symbols}
A direct computation shows that:
\[
\begin{aligned}
\Gamma^1_{11} &= \frac{1}{2D}\Bigl[G\,E_u - F\bigl(2F_u - E_v\bigr)\Bigr],\\[1mm]
\Gamma^1_{12} = \Gamma^1_{21} &= \frac{1}{2D}\Bigl[G\,E_v - F\,G_u\Bigr],\\[1mm]
\Gamma^1_{22} &= \frac{1}{2D}\Bigl[G\bigl(2F_v - G_u\bigr) - F\,G_v\Bigr],\\[1mm]
\Gamma^2_{11} &= \frac{1}{2D}\Bigl[E\bigl(2F_u - E_v\bigr) - F\,E_u\Bigr],\\[1mm]
\Gamma^2_{12} = \Gamma^2_{21} &= \frac{1}{2D}\Bigl[E\,G_u - F\,E_v\Bigr],\\[1mm]
\Gamma^2_{22} &= \frac{1}{2D}\Bigl[E\,G_v - F\bigl(2F_v - G_u\bigr)\Bigr],
\end{aligned}
\]
where \(D = EG - F^2\).
We note that the Christoffel symbols is not a tensor, but it is a connection that allows us to differentiate vectors along a curve on the surface.

The geodesic equations in these coordinates are given by
\[
\ddot{q}^{\sigma} + \Gamma_{\mu\nu}^{\sigma} \dot{q}^{\mu} \dot{q}^{\nu} = 0,
\]
where we set \(q^1=u\) and \(q^2=v\). Thus, the geodesic equations for each component are:
\begin{equation}
    \ddot{u} + \Gamma^1_{11}\,\dot{u}^2 + 2\,\Gamma^1_{12}\,\dot{u}\dot{v} + \Gamma^1_{22}\,\dot{v}^2 = 0
\end{equation}
\begin{equation}
    \ddot{v} + \Gamma^2_{11}\,\dot{u}^2 + 2\,\Gamma^2_{12}\,\dot{u}\dot{v} + \Gamma^2_{22}\,\dot{v}^2 = 0
\end{equation}

For the case where $F = 0$, the Christoffel symbols simplify to
\[
\begin{aligned}
    \Gamma_{11}^{1} &= \frac{E_{u}}{2E} &\quad \Gamma_{12}^{1} &= \Gamma_{21}^{1} = \frac{E_{v}}{2E} &\quad \Gamma_{22}^{1} &= -\frac{G_{u}}{2E} \\[4mm]
    \Gamma_{11}^{2} &= -\frac{E_{v}}{2G} &\quad \Gamma_{12}^{2} &= \Gamma_{21}^{2} = \frac{G_{u}}{2G} &\quad \Gamma_{22}^{2} &= -\frac{G_{v}}{2G}.
\end{aligned}
\]

Substituting these expressions into the geodesic equations, we obtain
\begin{equation}
    \ddot{u} + \frac{E_{u}}{2E}\,\dot{u}^2 + \frac{E_{v}}{2E}\,\dot{u}\dot{v} - \frac{G_{u}}{2E}\,\dot{v}^2 = 0
\end{equation}
\begin{equation}
    \ddot{v} - \frac{E_{v}}{2G}\,\dot{u}^2 + \frac{G_{u}}{2G}\,\dot{u}\dot{v} - \frac{G_{v}}{2G}\,\dot{v}^2 = 0
\end{equation}

\subsection{Geodesics on Surfaces of Revolution}

We now move our focus to surfaces of revolution.
Due to the complexity of the geodesic equations, we can use the symmetry of these surfaces to simplify the geodesic equations significantly.
We will introduce Clairaut's relation, which is a conserved quantity that arises from the symmetry of the surface.

\subsubsection{Clairaut's Relation}

\begin{definition}[Clairaut's Relation]
Let \(S\) be a surface of revolution, then along any geodesic on \(S\), the following relation holds:
\[
R\sin\phi = C
\]
\end{definition}
Here, \(R\) is the distance from the axis of revolution, \(\phi\) is the angle between the $\dot{\gamma}$ and the tangent vector to the geodesic, and \(C\) is a constant.

We say a Surface $S$ is $v$-Clairaut parametrized if $E_{u} = G_{u} = 0$ and $u$-Clairaut parametrized if $E_{v} = G_{v} = 0$.
Thus the geodesic equations for each Clairaut parametrization are given by
\begin{align*}
    \ddot{u} + \frac{E_{u}}{2E}\,\dot{u}^2 - \frac{G_{u}}{2E}\,\dot{v}^{2} &= 0 \qquad (u\text{-Clairaut Geodesic Equations)} \\
    \ddot{v} + \frac{G_{u}}{G}\,\dot{u}\dot{v} &= 0
\end{align*}
\begin{align*}
    \ddot{u} + \frac{E_{v}}{E}\,\dot{u}\dot{v} &= 0 \qquad (v\text{-Clairaut Geodesic Equations)} \\
    \ddot{v} - \frac{E_{v}}{2G}\,\dot{u}^2 + \frac{G_{u}}{2G}\,\dot{v}^2 &= 0
\end{align*}

Given we know that for a surface of revolution parametrized by Eq \eqref{eq:surface_of_revolution}, the metric tensor is given by
\[
    ds^2 = \left( x_{u}^2 + z_u^2 \right) du^2 + x^2 dv^2,
\]
where $E = x_{u}^2 + z_u^2$, $F = 0$, and $G = x^2$.
We compute the derivatives of the components of the metric tensor:
\[
    E_u = 2 x_u x_{uu} + 2 z_u z_{uu}, \quad E_v = 0, \quad G_u = 2 x x_u, \quad G_v = 0.
\]
Given $E_v = G_v = 0$, we thus have a $u$-Clairaut parametrization.
So Clairaut's relation can be written as
\begin{equation} \label{eq:Clairaut_relation_u}
    x(u) \sin(\phi) = \sqrt{G} \sin(\phi) = C
\end{equation}
The geodesic equations for a surface of revolution are given by
\begin{align*}
    \ddot{u} + \frac{x_u x_{uu} + z_u z_{uu}}{x_{u}^2 + z_u^2}\,\dot{u}^2 - \frac{x x_u}{x_{u}^2 + z_u^2}\,\dot{v}^{2} &= 0 \\
    \ddot{v} + \frac{x}{x}\,\dot{u}\dot{v} &= 0
\end{align*}

Solving the second equation by separating variables
\begin{align*}
    \frac{\ddot{v}}{\dot{u}\dot{v}} &= -\frac{G_{u}}{G}\dot{u} \\
    \int \frac{\ddot{v}}{\dot{u}\dot{v}} \dd{t} &= -\int \frac{G_{u}}{G}\dot{u} \dd{t} \\
    \ln(\dot{v}) &= -\ln(G) + c \\
    \dot{v} &= \frac{c}{G}
\end{align*}

Let us consider $\gamma(t)$ to be a unit speed geodesic curve, so that the unit speed relation thus gives us
\begin{align*}
    1 &= E \dot{u}^2 + G \dot{v}^2 \\
    1 &= E \dot{u}^2 + \frac{c^2}{G} \\
    \dot{u}^2 &= \frac{G - c^2}{EG} \\
    \dot{u} &= \pm \sqrt{\frac{G - c^2}{EG}} \\
\end{align*}

Dividing $\dot{v}$ by $\dot{u}$, we obtain
\[
    \dv{v}{u} = \frac{\dot{v}}{\dot{u}} = \pm \frac{c\sqrt{E}}{\sqrt{G}\sqrt{G - c^{2}}} 
\]
Integrating this expression gives us
\begin{equation} \label{eq:v_Clairaut_integral}
    v = \pm \int \frac{c\sqrt{E}}{\sqrt{G}\sqrt{G - c^{2}}} \dd{u}
\end{equation}

which serves to characterize geodesics for a $u$-Clairaut parametrization

If the surface is v-Clairaut parametrized, we can use the same steps to obtain
\[
    \dv{u}{v} = \pm \frac{c\sqrt{G}}{\sqrt{E}\sqrt{E - c^{2}}}
\]
Integrating this expression gives us
\begin{equation} \label{eq:u_Clairaut_integral}
    u = \pm \int \frac{c\sqrt{G}}{\sqrt{E}\sqrt{E - c^{2}}} \dd{v}
\end{equation}
This integral can be used to characterize geodesics for a $v$-Clairaut parametrization.
Clairaut's relation will then be given by
\begin{equation} \label{eq:Clairaut_relation_v}
    z(u) \sin(\phi) = \sqrt{E} \sin(\phi) = C
\end{equation}


\section{Geodesics of different Surfaces of Revolution}

\subsection{The Sphere}

The sphere is a surface of revolution obtained by rotating a circle about an axis.
The parametric representation of the sphere is given by
\[
    \mathbf{r}(u, v) = \begin{bmatrix} R \cos (u) \cos (v) \\ R \cos (u) \sin (v) \\ R \sin (u) \end{bmatrix},
\]
where \(R\) is the radius of the sphere, \(u = \theta\) is the angle of rotation about the \(z\)-axis, and \(v = \phi\) is the angle of rotation about the \(x\)-axis.
The tangent vectors are given by
\[
    \mathbf{r}_u = \begin{bmatrix} -R \sin (u) \cos (v) \\ -R \sin (u) \sin (v) \\ R \cos (u) \end{bmatrix}, \quad \mathbf{r}_v = \begin{bmatrix} -R \cos (u) \sin (v) \\ R \cos (u) \cos (v) \\ 0 \end{bmatrix}.
\]
The metric tensor for the sphere is given by
\[
    ds^2 = R^2 du^2 + R^2 \sin^2 (u) dv^2,
\]
Where \(E = R^2 \), \(F = 0\), and \(G = R^2 \sin^2 (u)\).
Thus the geodesic equations for the sphere are given by
\begin{align*}
    \ddot{u} + \sin(v) \cos(v) \dot{v}^2 &= 0 \\
    \ddot{v} + 2\tan(v) \dot{u} \dot{v} &= 0 
\end{align*}

The resulting system is a highly complex set of nonlinear differential equations, and in practice, geodesic equations are seldom solved directly.

We can use Clairaut's relation to simplify the geodesic equations.
Using Eq \eqref{eq:v_Clairaut_integral}, we get 
\[
    v = \pm \int \frac{c}{\sin(u)\sqrt{R^2\sin^{2}(u) - c^{2}}} \dd{u}
\]
These steps will be left out as they are not needed to understand the geodesic equations and are somewhat tedious. 
They can be found in full in \cite{Oprea2007}, (Note: \cite{Oprea2007} uses $v$-Clairaut parametrization, so the integration steps will slightly differ).
\[
\sin(v - d) = \lambda \cot(u)
\]
where \(d\) is a constant of integration and \(\lambda = \frac{c}{\sqrt{R^2 - c^2}}\).

Using the trigonometric identity \(\sin(v - d) = \sin(v)\cos(d) - \cos(v)\sin(d)\),  and find a common denominator $\sin(u)$ to get
\[
    \frac{\sin(v)\sin(u)}{\sin(u)}\cos(d) - \frac{\cos(v)\sin(u)}{\sin(u)}\sin(d) - \lambda \frac{\cos(u)}{\sin(u)} = 0
\]
If we substitution \(x = \cos(v)\sin(u), \quad y = \sin(v)\sin(u), \quad z = \cos(u)\), we can rewrite the equation as
\[
    y\cos(d) - x\sin(d) - \lambda z = 0
\]
Hence, the geodesic equations imply that the geodesics on the sphere lie on a plane \( ax + by + cz = d \) that passes through the origin.
So the geodesics on the sphere are great circles.

\subsection{The Hyperboloid}

The hyperboloid is a surface of revolution obtained by rotating a hyperbola about an axis.
The parametric representation of the hyperboloid (also called a ”Hyperboloid of one sheet”) is given by
\[
    \mathbf{r}(u, v) = \begin{bmatrix} a \cosh (u) \cos (v) \\ a \cosh (u) \sin (v) \\ b \sinh (u) \end{bmatrix},
\]
where \(a\) and \(b\) are the semi-major and semi-minor axes of the hyperbola, respectively.

The tangent vectors are given by
\[
    \mathbf{r}_u = \begin{bmatrix} a \sinh (u) \cos (v) \\ a \sinh (u) \sin (v) \\ b \cosh (u) \end{bmatrix}, \quad \mathbf{r}_v = \begin{bmatrix} -a \cosh (u) \sin (v) \\ a \cosh (u) \cos (v) \\ 0 \end{bmatrix}.
\]
The metric for the hyperboloid is given by
\[
    ds^2 = ((a^2 + b^2)\sinh^{2}(u) + b^2) du^2 + a^2 \cosh^{2}(u) dv^2
\]
where \(E = (a^2 + b^2)\sinh^{2}(u) + b^2\), \(F = 0\), and \(G = a^2 \cosh^{2}(u)\).
Taking the derivatives of the components of the metric tensor, we find
\[
    E_u = 2(a^2 + b^2)\sinh(u)\cosh(u), \quad E_v = 0, \quad G_u = 2a^2 \sinh(u) \cosh(u), \quad G_v = 0.
\]
Thus, the hyperboloid is a $u$-Clairaut parametrization.
So the geodesic equations for the hyperboloid are given by
\begin{align*}
    \ddot{u} + \frac{(a^2 + b^2)\sinh(u)\cosh(u)}{(a^2 + b^2)\sinh^{2}(u) + b^2}\,\dot{u}^2 - \frac{a^2 \sinh(u) \cosh(u)}{(a^2 + b^2)\sinh^{2}(u) + b^2}\,\dot{v}^{2} &= 0 \\
    \ddot{v} + \tanh(u) \dot{u} \dot{v} &= 0
\end{align*}
Using Clairaut's relation, we can simplify the geodesic equations.
Using Eq \eqref{eq:v_Clairaut_integral}, we get
\[
    v = \pm \int \frac{c\sqrt{(a^2 + b^2)\sinh^2(u) + b^2}}{a\cosh(u)\sqrt{a^2\cosh^2(u) - c^2}}\,\dd{u}
\]
We use Clairaut's relation to analyze the geodesics on the hyperboloid.
\[
    a \cosh(u) \sin(\phi) = C  
\]
By assuming that $u$ is very small number, therefore we will consider into 3 cases:
\begin{itemize}
    \item If $C = 0$, then $\phi = 0$ and the geodesic corresponds to a meridian on the hyperboloid.
    \item If $C = a$ at $u = u_0 = 0$, then $\phi = \frac{\pi}{2}$ and the geodesic a parallel on the hyperboloid.
    \item If $\abs{C} < \abs{a}$, then $\phi$ is a constant angle and the geodesic is a helix on the hyperboloid.
\end{itemize}

\subsection{The Torus}

The torus is a surface of revolution obtained by rotating a circle about an axis that is not in the plane of the circle.
The parametric representation of the torus is given by
\[
    \mathbf{r}(u, v) = \begin{bmatrix} (R + r \cos (u)) \cos (v) \\ (R + r \cos (u)) \sin (v) \\ r \sin (u) \end{bmatrix},
\]
where \(R\) is the distance from the center of the torus to the center of the tube, and \(r\) is the radius of the tube.
The tangent vectors are given by
\[
    \mathbf{r}_u = \begin{bmatrix} -r \sin (u) \cos (v) \\ -r \sin (u) \sin (v) \\ r \cos (u) \end{bmatrix}, \quad \mathbf{r}_v = \begin{bmatrix} -(R + r \cos (u)) \sin (v) \\ (R + r \cos (u)) \cos (v) \\ 0 \end{bmatrix}.
\]
The metric for the torus is given by
\[
    ds^2 = r^2 du^2 + (R + r \cos (u))^2 dv^2,
\]
where \(E = r^2\), \(F = 0\), and \(G = (R + r \cos (u))^2\).
Taking the derivatives of the components of the metric tensor, we find
\[
    E_u = 0, \quad E_v = 0, \quad G_u = -2(R + r \cos (u))r \sin (u), \quad G_v = 0.
\]

Thus, the torus is a $u$-Clairaut parametrization.
So the geodesic equations for the torus are given by
\begin{align*}
    \ddot{u} + \frac{(R + r \cos (u))r \sin (u)}{r^2}\,\dot{u}^2 - \frac{(R + r \cos (u))^2}{r^2}\,\dot{v}^{2} &= 0 \\
    \ddot{v} + \frac{r \sin (u)}{R + r \cos (u)}\,\dot{u}\dot{v} &= 0
\end{align*}

Using Clairaut's relation, we can simplify the geodesic equations.
Using Eq \eqref{eq:v_Clairaut_integral}, we get
\[
    v = \pm \int \frac{cr}{(R + r \cos (u))\sqrt{(R + r \cos (u))^2 - c^2}}\,\dd{u}
\]
We can use Clairaut's relation to analyze the geodesics on the torus.
\[
    (R + r \cos (u)) \sin(\phi) = C
\]
By assuming that $u$ is very small number, therefore we will consider into 3 cases:

\begin{itemize}
    \item If $C = 0$, then $\phi = 0$ and the geodesic corresponds to a meridian on the torus.
    \item If $C = R + r$ at $u = u_0 = 0$, then $\phi = \frac{\pi}{2}$ and the geodesic a parallel on the torus.
    \item If $\abs{C} < \abs{R + r}$, then $\phi$ is a constant angle and the geodesic is a helix on the torus.
    % \item If $\abs{C} > \abs{R + r}$, then the geodesic is a closed curve on the torus.
\end{itemize}

\section{Symmetries and Isometries}

\subsection{Groups}

\subsection{Noether's Theorem}

\subsection{Killing Vectors}

\subsection{Jacobi Fields}

Might touch on this, idk.

\section{Applications to Black Holes}

\subsection{The Schwarzschild Metric}

\subsection{Null Geodesics}

\section{A Look at Minkowski Spacetime}

% refereces
\section{References}

% get the references from the bibtex file
\bibliographystyle{plain}
\bibliography{references}

\section{Appendix}

\appendix

Probably not needed, but if I need to put my sagemath code here or any other derivations, I can do it here.

\end{document}