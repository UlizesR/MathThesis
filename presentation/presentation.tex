\documentclass[12pt]{beamer}
\usepackage{amsmath, amssymb, amsthm}
\usepackage{physics}
\usetheme{Madrid}

\title[Geodesics on Surfaces]{Geodesics on Surfaces of Revolution and Ruled Surfaces: \\ A Study of Symmetries and Isometries}
\author{Ulizes Raudales}
\date{\today}

\begin{document}

% Slide 1: Title Slide
\begin{frame}
  \titlepage
\end{frame}

% Slide 3: Introduction
\section{Introduction}
\begin{frame}
  \frametitle{Introduction}
  \begin{itemize}
    \item \textbf{Motivation:} I like differential geometry and group theory.
    \item \textbf{Objective:} Study geodesics on surfaces with rotational and translational symmetries.
    \item \textbf{Importance:} Geodesics are fundamental to understanding curved spaces.
    \item \textbf{Key Question:} How do inherent symmetries simplify the geodesic equations?
  \end{itemize}
\end{frame}

% Slide 4: Thesis Focus and Objectives
\begin{frame}
  \frametitle{Thesis Focus and Objectives}
  \begin{itemize}
    \item \textbf{Derivation:} Formulate geodesic equations via variational calculus and the Euler--Lagrange framework.
    \item \textbf{Analysis:} Examine specific examples (sphere, cylinder) to illustrate geodesic behavior.
    \item \textbf{Symmetry:} Use rotational and translational symmetries to identify conserved quantities.
    \item \textbf{Future Work:} Extend analysis to more complex ruled surfaces and explore numerical simulations.
  \end{itemize}
\end{frame}

% Slide 5: Background: Differential Geometry & Tensor Notation
\section{Background and Preliminaries}
\begin{frame}
  \frametitle{Differential Geometry}
  \begin{itemize}
    \item \textbf{Manifolds:}
      \begin{itemize}
        \item $n$-dimensional topological spaces that locally resemble Euclidean space.
        \item Examples: sphere, torus, cylinder, Minkowski spacetime, Klein bottle.
      \end{itemize}
    \item We consider surfaces (2-manifolds) embedded in \(\mathbb{R}^3\).
    \item We define the parameterization of a surface as \(\mathbf{r}(u,v)\).
    \item \(\mathbf{r}: \mathbb{R}^2 \to \mathbb{R}^3\) is a smooth map where \(u(t),v(t) \in \mathbb{R}^{2}\) are local coordinates.
  \end{itemize}
\end{frame}

\begin{frame}
    \frametitle{Differential Geometry: Metric Tensor}
    \footnotesize
    \begin{itemize}
      \item \textbf{Tangent Space:} 
        \[
        T_{p}(M) = \mathrm{span}\left\{\pdv{\mathbf{r}}{u},\pdv{\mathbf{r}}{v}\right\}.
        \]
        % \pause\vspace{0.3cm}
        \item \textbf{Metric Tensor:} 
        Let \(\mathbf{r}_u = \pdv{\mathbf{r}}{u}\) and \(\mathbf{r}_v = \pdv{\mathbf{r}}{v}\).
        \[
        g_{\mu\nu} = \mathbf{r}_{\mu} \cdot \mathbf{r}_{\nu} \quad \text{where} \quad \mu,\nu \in \{u,v\}.
        \]
        % \pause\vspace{0.3cm}
      \item Let 
        \[
        E = \mathbf{r}_u \cdot \mathbf{r}_u,\quad F = \mathbf{r}_u \cdot \mathbf{r}_v,\quad G = \mathbf{r}_v \cdot \mathbf{r}_v.
        \]
        \[
            E_{u} = \pdv{E}{u}, \quad E_{v} = \pdv{E}{v}, \quad F_{u} = \pdv{F}{u}, \quad F_{v} = \pdv{F}{v}, \quad G_{u} = \pdv{G}{u}, \quad G_{v} = \pdv{G}{v}.
        \]
        % \pause\vspace{0.3cm}
      \item \textbf{Metric:} 
        \[
        ds^2 = E\,du^2 + 2F\,du\,dv + G\,dv^2.
        \]
    \end{itemize}
  \end{frame}

\begin{frame}
  \frametitle{Derivation of Geodesic Equations (I)}
  \footnotesize
  \begin{itemize}
    \item \textbf{Action Functional:}
      \[
      S[\gamma] = \int_{a}^{b} \mathcal{L}(t, q^\lambda, \dot{q}^\lambda)\, dt.
      \]
    \item \textbf{Euler--Lagrange Equations:} Assuming 
      \[
      \mathcal{L} = g_{\mu\nu}\,\dot{q}^\mu\dot{q}^\nu,
      \]
      they become
      \[
      \frac{d}{dt}\!\left(\frac{\partial \mathcal{L}}{\partial \dot{q}^\lambda}\right) - \frac{\partial \mathcal{L}}{\partial q^\lambda} = 0.
      \]
    \item \textbf{Geodesic Equation:} Thus,
      \[
      \dv[2]{q^\lambda}{t} + \Gamma^\lambda_{\mu\nu}\,\dv{q^\mu}{t}\dv{q^\nu}{t} = 0.
      \]
  \end{itemize}
\end{frame}

\begin{frame}
  \frametitle{Derivation of Geodesic Equations (II)}
  \footnotesize
  \begin{itemize}
    \item \textbf{Christoffel Symbols:} Encode the curvature of the manifold.
      \[
      \begin{aligned}
      \Gamma^u_{uu} &= \frac{1}{2D}\Bigl[G\,E_u - F\bigl(2F_u - E_v\bigr)\Bigr], \quad \Gamma^u_{uv} = \Gamma^u_{vu} = \frac{1}{2D}\Bigl[G\,E_v - F\,G_u\Bigr],\\[1mm]
      \Gamma^u_{vv} &= \frac{1}{2D}\Bigl[G\bigl(2F_v - G_u\bigr) - F\,G_v\Bigr], \quad \Gamma^v_{uu} = \frac{1}{2D}\Bigl[E\bigl(2F_u - E_v\bigr) - F\,E_u\Bigr],\\[1mm]
      \Gamma^v_{uv} = \Gamma^v_{vu} &= \frac{1}{2D}\Bigl[E\,G_u - F\,E_v\Bigr], \quad \Gamma^v_{vv} = \frac{1}{2D}\Bigl[E\,G_v - F\bigl(2F_v - G_u\bigr)\Bigr],
      \end{aligned}
      \]
    \item \textbf{Geodesic Equations for \(u\) and \(v\):}
      \begin{align}
        \ddot{u} + \Gamma^u_{uu}\,\dot{u}^2 + 2\,\Gamma^v_{uv}\,\dot{u}\dot{v} + \Gamma^u_{vv}\,\dot{v}^2 &= 0, \label{eq:geo1}\\[1ex]
        \ddot{v} + \Gamma^v_{uu}\,\dot{u}^2 + 2\,\Gamma^v_{uv}\,\dot{u}\dot{v} + \Gamma^v_{vv}\,\dot{v}^2 &= 0. \label{eq:geo2}
      \end{align}
  \end{itemize}
\end{frame}

% Slide 7: Surfaces of Revolution
\section{Surfaces of Revolution}
\begin{frame}
  \frametitle{Surfaces of Revolution}
  \footnotesize
  \begin{itemize}
    \item \textbf{Definition:} Obtained by rotating a profile curve about an axis.
    \item \textbf{Parametrization:}
      \[
      \mathbf{r}(u,v) = \begin{bmatrix} x(v)\cos(u) \\ x(v)\sin(u) \\ z(v) \end{bmatrix},
      \]
      where: \(u\) is the angular coordinate and \(v\) parameterizes the profile curve.
    \item \textbf{Metric Components:}
      \[
      E = x(v)^2,\quad F = 0,\quad G = \dot{x}(v)^2 + \dot{z}(v)^2.
      \]
      \item \textbf{Geodesic Equations:}
      The geodesic equations for this surface are
      \begin{align*}
    \ddot{u} + \frac{2\dot{x}}{x}\,\dot{v}\,\dot{u} = 0,\\
    \ddot{v} - \frac{x\dot{x}}{\dot{x}^2 + \dot{z}^2}\,\dot{u}^2 + \frac{\ddot{x}\,\dot{x} + \ddot{z}\,\dot{z}}{\dot{x}^2 + \dot{z}^2}\,\dot{v}^2 = 0.
\end{align*}
    \item \textbf{Advantage:} Rotational symmetry yields conserved quantities simplifying the equations.
  \end{itemize}
\end{frame}

% Slide 8: Special Curves: Meridians and Parallels
\begin{frame}
  \frametitle{Meridians and Parallels}
  \begin{itemize}
    \item \textbf{Meridians:} 
      \begin{itemize}
         \item Defined by constant \(u\) (\(\dot{u}=0\)); reduce to a linear equation in \(v\).
      \end{itemize}
    \item \textbf{Parallels:} 
      \begin{itemize}
         \item Defined by constant \(v\); geodesic only if \(x'(v_0)=0\) (local extremum of \(x(v)\)).
      \end{itemize}
    \item \textbf{Interpretation:} Illustrate how symmetry reduces the complexity of geodesic equations.
  \end{itemize}
\end{frame}

% Slide 9: Detailed Examples: Sphere and Cylinder
\begin{frame}
  \frametitle{Examples: Sphere and Cylinder}
  \textbf{Sphere:}
  \begin{itemize}
    \item \textbf{Parametrization:}
      \[
      \mathbf{r}(\theta,\phi)=\begin{bmatrix} R\cos\theta\sin\phi \\ R\sin\theta\sin\phi \\ R\cos\phi \end{bmatrix}.
      \]
    \item \textbf{Metric:}
      \[
      ds^2 = R^2\sin^2\phi\, d\theta^2 + R^2\,d\phi^2.
      \]
    \item \textbf{Geodesics:} 
    \[
    \begin{aligned}
        \ddot{\theta} &- 2\tan \theta\,\dot{\phi}\,\dot{\theta} = 0,\\
        \ddot{\phi} &- \cos \theta \sin \theta\,\dot{\theta}^2  = 0.
    \end{aligned}
    \]
    Great circles represent the shortest paths.
  \end{itemize}
\end{frame}

\begin{frame}
    \textbf{Cylinder:}
    \textbf{Cylinder:}
    \begin{itemize}
      \item \textbf{Parametrization:}
        \[
        \mathbf{r}(u,v)=\begin{bmatrix} R\cos u \\ R\sin u \\ v \end{bmatrix}.
        \]
      \item \textbf{Metric:}
        \[
        ds^2 = R^2\,du^2 + dv^2.
        \]
        \item \textbf{Geodesics:} 
        \[
            \ddot{u} = 0, \quad \ddot{v} = 0.
        \]
        \[
            u = at + b, \quad v = ct + d.
        \]
        Helices along the cylinder's surface.
    \end{itemize}
\end{frame}

% Slide 10: Ruled Surfaces and Their Geodesics
\section{Ruled Surfaces and Their Geodesics}
\begin{frame}
  \frametitle{Ruled Surfaces}
  \begin{itemize}
    \item \textbf{Definition:} Generated by moving a straight line (ruling) along a base curve.
    \item \textbf{Parametrization:}
      \[
      \mathbf{r}(u,v) = \gamma(u) + v\,\mathbf{d}(u),
      \]
      where:
      \begin{itemize}
         \item \(\gamma(u)\) is the base curve.
         \item \(\mathbf{d}(u)\) is the direction vector.
      \end{itemize}
    \item \textbf{Analysis:} Lacks full rotational symmetry, making the geodesic analysis more complex.
    \item \textbf{Future Work:} Focus on exploiting partial symmetries to simplify the analysis.
  \end{itemize}
\end{frame}

% Slide 11: Symmetries and Isometries
\begin{frame}
  \frametitle{Symmetries and Isometries}
  \begin{itemize}
    \item \textbf{Symmetries:}
      \begin{itemize}
         \item Provide conserved quantities (e.g., Clairaut's relation).
         \item Lower the order of geodesic differential equations.
         \item Focus on Lie groups and algebras and their actions.
         \item $SO(3)$ for rotational symmetry, $SE(3)$ for translational symmetry.
      \end{itemize}
    \item \textbf{Isometries:}
      \begin{itemize}
         \item Transformations that preserve distances.
         \item Enable mapping of complex problems to simpler, equivalent ones.
      \end{itemize}
    \item \textbf{Outcome:} Both are key to understanding and simplifying geodesic behavior.
  \end{itemize}
\end{frame}

% % Slide 12: Conclusion and Future Work
% \begin{frame}
%   \frametitle{Conclusion and Future Work}
%   \begin{itemize}
%     \item \textbf{Summary:}
%       \begin{itemize}
%          \item Derived geodesic equations using variational methods.
%          \item Analyzed surfaces of revolution (e.g., sphere, cylinder) with clear examples.
%          \item Introduced the concept of ruled surfaces for further study.
%       \end{itemize}
%     \item \textbf{Future Directions:}
%       \begin{itemize}
%          \item Extend analysis to more general ruled surfaces.
%          \item Develop numerical simulations for complex geodesic equations.
%          \item Apply findings in related fields like physics and computer graphics.
%       \end{itemize}
%     \item \textbf{Final Thought:} Symmetry and isometry are central to unlocking the geometry of curved spaces.
%   \end{itemize}
% \end{frame}

% Slide 13: Challenges and Open Questions
\begin{frame}
  \frametitle{Challenges and Open Questions}
  \begin{itemize}
    \item \textbf{Analytical Complexity:} How can we further simplify the geodesic equations on less symmetric surfaces?
    \item \textbf{Numerical Approaches:} What are the most effective numerical methods to simulate geodesics on complex surfaces?
    \item \textbf{Extension to Higher Dimensions:} Can these techniques be generalized to manifolds beyond \(\mathbb{R}^3\)?
    \item \textbf{Interdisciplinary Applications:} How can the study of geodesics impact fields such as computer graphics and physical modeling?
  \end{itemize}
\end{frame}

% Slide 14: References
\begin{frame}
    \frametitle{References}
    \begin{itemize}
      \item Rigatti, Olivia Grace. \emph{Characterizing Geodesics on Surfaces of Revolution}. Whitman College, 2023. :contentReference[oaicite:0]{index=0}
      \item Ramirez, Steven John. \emph{Geodesics of Ruled Surfaces}. California State University, San Bernardino, 2001. :contentReference[oaicite:1]{index=1}
      \item \emph{Supplemental Lecture 4: Surfaces and Euclidean, Spherical and Hyperbolic Geometry}. :contentReference[oaicite:2]{index=2}
      \item Oprea, John. \emph{Differential Geometry and Its Applications} (2nd ed.). The Mathematical Association of America, 2007. :contentReference[oaicite:3]{index=3}
    \end{itemize}
\end{frame}

% Slide 14: Questions and Discussion
\begin{frame}
  \frametitle{Questions and Discussion}
  \begin{center}
    \Large{Thank you!}
    
    \vspace{0.5cm}
    \Large{Any Questions?}
  \end{center}
\end{frame}

\end{document}
