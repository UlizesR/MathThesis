% Beamer presentation
\documentclass{beamer}

\usepackage{graphicx}
\usepackage{amsmath}
\usepackage{amssymb}
\usepackage{amsfonts}
\usepackage{amsthm}
\usepackage{booktabs}
\usepackage{listings}
\usepackage{xcolor}
\usepackage{physics}

\usetheme{Madrid}
\useoutertheme{miniframes}
\useinnertheme{circles}

\definecolor{UBCblue}{rgb}{0.04706, 0.13725, 0.26667} % UBC Blue (primary)
\usecolortheme[named=UBCblue]{structure}

\newcommand{\R}{\mathbb{R}}

\begin{document}

\begin{frame}
	\title{A look at geodesics through symmetries and their applications}
	\author{Uli Raudales}
	\date{\today}
	\titlepage
\end{frame}

\begin{frame}{Outline}
  	\tableofcontents
\end{frame}

\section{Preliminaries}
\begin{frame}{Preliminaries}
	\footnotesize
	\begin{itemize}
		\item \textbf{Differential geometry:} Study of smooth topological spaces called \emph{manifolds} using tools from calculus and linear algebra.
	\end{itemize}
\end{frame}

\section{Surfaces and Curves}
\begin{frame}{Surfaces}
	% \footnotesize
	\begin{definition}[Surface]\label{def:surface}
		A regular surface $\mathcal{S}\subset\R^3$ is the image of a one-to-one smooth map
		\[
			\mathbf{r}: U\subset\R^2 \to \R^3,\quad
			\mathbf{r}(u,v) = \bigl\langle x(u,v),\,y(u,v),\,z(u,v)\bigr\rangle,
		\]
		such that $\mathbf{r}_u \times \mathbf{r}_v \neq \mathbf{0} \quad\text{for all }(u,v)\in U $.
	\end{definition}
	\hfill
	\vspace{1ex}
	\begin{itemize}[<+->]
		\item A surface is a 2D manifold embedded in $\R^3$, locally resembling $\R^2$.
		\item We write $\mathbf{r}_u = \frac{\partial\mathbf{r}}{\partial u}\big|_{(u_0,v_0)}$ and $\mathbf{r}_v = \frac{\partial\mathbf{r}}{\partial v}\big|_{(u_0,v_0)}$ for the tangent vectors at a point $\mathbf{r}(u_0,v_0)$.
	\end{itemize}
\end{frame}

\begin{frame}{The Metric}
	% \footnotesize
	\begin{definition}[Metric]\label{def:metric}
		A \emph{metric} on a surface $\mathcal{S}$ is a is a smoothly varying inner product on the tangent plane at each point of the surface.
		\[
			ds^2 = E\,du^2 + 2F\,du\,dv + G\,dv^2,
		\]
		where $E = \mathbf{r}_u\cdot\mathbf{r}_u$, $F = \mathbf{r}_u\cdot\mathbf{r}_v$, and $G = \mathbf{r}_v\cdot\mathbf{r}_v$.
	\end{definition}
	\hfill 
	\pause 
	\begin{itemize}[<+->]
		\item The metric defines lengths, angles, and areas on the surface.
		\item The coefficients $E$, $F$, and $G$ are called the \emph{first fundamental form} of the surface often characterized as:
		\[
			g_{\mu\nu} = \mathbf{r}_\mu \cdot \mathbf{r}_\nu =
			\begin{bmatrix}
				E & F \\
				F & G
			\end{bmatrix} 
		\] 
	\end{itemize}
\end{frame}

\begin{frame}{Curves}
	% \footnotesize
	\begin{definition}[Curve]\label{def:curve}
		A \emph{curve} $\gamma$ on a surface $\mathcal{S}$ is a smooth map $\gamma: I \to \mathcal{S}$, where $I$ is an interval in $\R$.
		\[
		    \gamma(t) = \mathbf{r}(u(t),v(t))
		\]
	\end{definition}
	\vspace{1ex}
	\begin{itemize}[<+->]
		\item $\dot{\gamma}(t) = \frac{d\gamma}{dt} = \dot{u}(t)\mathbf{r}_u  + \dot{v}(t)\mathbf{r}_v$
		\item $\ddot{\gamma}(t) = \dv[2]{\gamma}{t} = \dot{u}(t)^{2}\mathbf{r}_{uu} + \dot{v}(t)^{2}\mathbf{r}_{vv} + 2\dot{u}(t)\dot{v}(t)\mathbf{r}_{uv} + \ddot{u}\mathbf{r}_{u} + \ddot{v}\mathbf{r}_{v}.$
		\item \textbf{Unit Speed Condition:} Under arc length re-parametrization, $\|\dot{\gamma}(t)\| = 1$.
		% \item<4-> \textbf{Curvature:} $\kappa_{\gamma}(t) = \frac{\|\ddot{\gamma}(t)\|}{\|\dot{\gamma}(t)\|^3} \implies \|\ddot{\gamma}(t)\|$.
		% \item<5-> \textbf{Normal Curvature:} $\kappa_{\mathbf{N}}(t) = \kappa_{\gamma}(t) \cdot \mathbf{N}(t)$, where $\mathbf{N}(t)$ is the unit normal vector to the surface at $\gamma(t)$.
	\end{itemize}	
\end{frame}

\begin{frame}{Curvature}
	% \footnotesize
	\begin{definition}[Curvature]\label{def:curvature}
		The \emph{curvature} of a curve defines how much the curve deviates from being a straight line.
		\[
			\kappa_{\gamma}(t) = \frac{\|\dot{\gamma}(t) \times \ddot{\gamma}(t)\|}{\|\dot{\gamma}(t)\|^3} \implies \|\ddot{\gamma}(t)\|
		\]
	\end{definition}
	\vspace{1ex}
	\begin{itemize}[<+->]
		\item \textbf{Normal Curvature:} $\kappa_{\mathbf{N}}(t) = \kappa_{\gamma}(t) \cdot \mathbf{N}(t)$, where $\mathbf{N}(t)$ is the unit normal vector to the surface at $\gamma(t)$.
		\item \textbf{Geodesic Curvature:} $\kappa_{g}(t) = \kappa_{\gamma}(t) - \kappa_{\mathbf{N}}(t)$, which measures the curvature of the curve relative to the surface.
	\end{itemize}
\end{frame}

\section{Geodesics}
\begin{frame}{Geodesics}
	\footnotesize
	\begin{definition}[Geodesic]\label{def:geodesic}
		A \emph{geodesic} is a curve $\gamma$ on a surface $\mathcal{S}$ that has zero geodesic curvature.
		\[
			\kappa_{g}(t) = 0
		\]
	\end{definition}
	\vspace{1ex}
	\begin{itemize}[<+->]
		\item Geodesics are the "straightest" paths on a curved surface.
		\item Often, geodesics can be considered as the shortest paths between two points on the surface.
		\item Geodesics are described by the geodesic equation:
		\[
		    \dv[2]{q^\sigma}{t} + \sum_{\mu,\nu=1}^{n} \Gamma_{\mu\nu}^\sigma \dv{q^\mu}{t} \dv{q^\nu}{t} = 0 \qquad \Gamma_{\mu\nu}^\sigma = \frac{1}{2}\sum_{\lambda=1}^{n}g^{\sigma\lambda}\left(\pdv{g_{\mu\nu}}{q^\lambda} + \pdv{g_{\nu\lambda}}{q^\mu} - \pdv{g_{\mu\lambda}}{q^\nu}\right)
		\]
		where $\Gamma_{\mu\nu}^\sigma$ are the Christoffel symbols of the second kind, describing how the coordinate system curves.
	\end{itemize}
\end{frame}

\section{A Look at Surfaces of Revolution}
\begin{frame}{Surfaces of Revolution}
	\footnotesize
	\begin{definition}[Surface of Revolution]\label{def:surface_of_revolution}
		A \emph{surface of revolution} is a surface generated by rotating a curve $\gamma$ about an axis. Under rotation about the $z$-axis, the surface can be described as:
		\[
			\mathcal{S} = \mathbf{r}(u,v) = \bigl\langle x(u)\cos(v),\,x(u)\sin(v),\,z(u)\bigr\rangle \qquad u\in I,\,v\in [0,2\pi]
		\]
	\end{definition}
	\vspace{1ex}
	\begin{itemize}[<+->]
		\item Surfaces of revolution are characterized by their symmetry about an axis.
		\item The metric on a surface of revolution is given by:	
		\[
		    ds^{2} \;=\; \big(x_{u}^{2} + z_{u}^{2}\big)\,du^{2} + x(u)^{2}\,dv^{2} \implies du^{2} + x(u)^{2}dv^{2} 
		\]
		\item By the unit speed condition, the following equation hold:
		\[
			1 = \left(\dv{u}{t}\right)^{2} + x(u)^{2}\left(\dv{v}{t}\right)^{2}
		\]
	\end{itemize}
\end{frame}

\begin{frame}{Geodesics on Surfaces of Revolution}
	\footnotesize
	\begin{itemize}[<+->]
		\item The geodesic equation for a surface where $F = 0$ is given by:
		\[
			\ddot{u} + \frac{E_{u}}{2E}\dot{u}^{2} + 2\frac{E_{v}}{2E}\dot{u}\dot{v} - \frac{G_{u}}{2E}\dot{v}^{2} = 0
		\]
		\[
			\ddot{v} - \frac{E_{v}}{2G}\dot{u}^{2} + 2\frac{G_{u}}{2G}\dot{u}\dot{v} + \frac{G_{v}}{2G}\dot{v}^{2} = 0
		\]
		Where $E_{u} = \pdv{E}{u}$, $E_{v} = \pdv{E}{v}$, $G_{u} = \pdv{G}{u}$, and $G_{v} = \pdv{G}{v}$.
		\item This corresponds to the geodesic equations for a surface of revolution:
	\end{itemize}
	\begin{align*}
		\ddot{u} + \frac{x_{u}x_{uu} + z_{u}z_{uu}}{ x_{u}^{2} + z_{u}^{2}}\,\dot{u}^2 - \frac{x x_{u}}{x_{u}^{2} + z_{u}^{2}}\,\dot{v}^{2} = 0 &\implies \ddot{u} - x x_{u}\,\dot{v}^{2} = 0 \\
		\ddot{v} - 2\frac{x_{u}}{x}\dot{u}\dot{v} = 0 &\implies \ddot{v} - 2\frac{x_{u}}{x}\dot{u}\dot{v} = 0
	\end{align*}
\end{frame} 
% 	\begin{itemize}[<+->]
% 	% \item For when $v = v_0$ is constant, meridians (lines of longitude) are geodesics
% 	% \[
% 	% 	\ddot{u} = 0 \implies u(t) = u_0 + \dot{u}_0 t
% 	% \]
% 	% \item For when $u = u_0$ is constant, parallels (lines of latitude) are geodesics iff $x(u_0) = 0$
% 	% \[
% 	% 	x \, x_{u} \, \dot{v} = 0 
% 	% \]
% \end{itemize}


\section{Symmetries \& Killing Fields}
\begin{frame}{Symmetries \& Isometries}
	\begin{definition}[Symmetry]
		A \emph{symmetry} of a surface $\mathcal{S}$ is a diffeomorphism $\phi: \mathcal{S} \to \mathcal{S}$ that preserves the structure of the surface.
		They are actions/transformations that leave the surface unchanged.
	\end{definition}
	\vspace{1ex}
	\pause 
	\begin{definition}[Isometry]
		An \emph{isometry} is a symmetry that preserves the metric, that is, it preserves lengths and angles:
		\[
			g(\phi(p), \phi(q)) = g(p, q)
		\]
	\end{definition}
\end{frame}

\section{Symmetries \& Killing Fields}
\begin{frame}{Example: Symmetries \& Isometries}
	\footnotesize
	\begin{definition}[Group]
		A \emph{group} is a set $G$ with a binary operation $\circ$ that satisfies the following properties:
		\begin{itemize}
			\item \textbf{Closure:} For all $a, b \in G$, $a \circ b \in G$.
			\item \textbf{Associativity:} For all $a, b, c \in G$, $(a \circ b) \circ c = a \circ (b \circ c)$.
			\item \textbf{Identity:} There exists an element $e \in G$ such that for all $a \in G$, $e \circ a = a \circ e = a$.
			\item \textbf{Inverse:} For each element $a \in G$, there exists an element $a^{-1} \in G$ such that $a \circ a^{-1} = e$.
		\end{itemize}
	\end{definition}

	\begin{itemize}[<+->]
		\item \textbf{Euclidean Group $E(n)$:} The group of all isometries of $\R^n$, including translations and rotations.
		\item \textbf{Orthogonal Group $O(n)$:} The group of all reflections and rotations in $\R^n$.
		\item \textbf{Poincaré Group:} The group of all isometries of Minkowski spacetime, including translations and Lorentz transformations.
		\item \textbf{Killing Vector Fields:} A group of isometries that generate symmetries of the metric. 
	\end{itemize}
\end{frame}

\begin{frame}{Killing Vector Fields}
	\footnotesize
	\begin{definition}[Killing Vector Field]
		A \emph{Killing vector field} is a vector field $K^\mu$ on a manifold $\mathcal{M}$ that satisfies the Killing equation:
		\[
			\nabla_\mu K_\nu + \nabla_\nu K_\mu = 0
		\]
		where $\nabla_\mu K_\nu = \pdv{K_\nu}{x^\mu} + \Gamma_{\mu\nu}^\lambda K_\lambda$ is the covariant derivative of the vector field.
	\end{definition}
	\vspace{1ex}
	\pause 
	\begin{itemize}[<+->]
		\item Killing vector fields generate symmetries of the metric, preserving distances and angles.
		\item Easiest way to find Killing vector fields is to look for symmetries of the metric, (i.e. the coordinates that the metric is independent of).
		\item The number of independent Killing vector fields is related to the number of symmetries of the manifold.
	\end{itemize}
\end{frame}

% 7. Schwarzschild Geodesics
\section{Black Hole Geodesics}
\begin{frame}{Schwarzschild Metric}
  The Schwarzschild solution describes the geometry of spacetime around a non-rotating, spherically symmetric mass:
  \[ds^2=-(1-2GM/r)dt^2+(1-2GM/r)^{-1}dr^2+r^2(d\theta^2 + \sin^2\theta\,d\phi^2),\]
  \begin{itemize}[<+->]
    \item The Schwarzschild radius $r_s = 2GM$ marks the event horizon of a black hole
    \item Time-translation symmetry $\to$ Killing vector $\partial_t$ $\to$ conserved energy $E$
    \item Rotational symmetry $\to$ Killing vector $\partial_\phi$ $\to$ conserved angular momentum $L$
    \item These symmetries significantly simplify the geodesic equations
  \end{itemize}
\end{frame}

\begin{frame}{Effective Potential}
  Using the conserved quantities, the geodesic motion can be reduced to a one-dimensional problem:
  \[ \tfrac12\dot{r}^2 + V_{\rm eff}(r)=\tfrac12E^2, \]
  \[ V_{\rm eff}(r)=(1-2GM/r)\left(\frac{L^2}{r^2}+\epsilon\right). \]
  \begin{itemize}[<+->]
    \item $\epsilon=0$ for massless particles (null geodesics, e.g., light)
    \item $\epsilon=1$ for massive particles (timelike geodesics, e.g., spacecraft)
    \item The effective potential combines gravitational effects and the centrifugal barrier
    \item The shape of $V_{\rm eff}(r)$ determines possible orbital trajectories:
      \begin{itemize}
        \item Bound orbits: $E^2 < V_{\rm eff}(r \to \infty)$
        \item Unbound orbits: $E^2 > V_{\rm eff}(r \to \infty)$
      \end{itemize}
  \end{itemize}
\end{frame}

\begin{frame}{Circular Orbits \& Photon Sphere}
  Circular orbits occur at the extrema of the effective potential, where $dV_{\rm eff}/dr=0$:
  \begin{itemize}[<+->]
    \item For null geodesics (light), there exists a unique unstable circular orbit at $r=3GM$
    \item This special radius defines the \emph{photon sphere}, where light can orbit the black hole
    \item For timelike geodesics (massive particles):
      \begin{itemize}
        \item The innermost stable circular orbit (ISCO) is at $r=6GM$
        \item Circular orbits at $r<6GM$ are unstable
        \item Circular orbits at $r>6GM$ are stable
      </itemize>
    \item These special orbits are critical for understanding accretion disks and gravitational lensing around black holes
  \end{itemize}
\end{itemize}
\end{frame}

\end{document}
